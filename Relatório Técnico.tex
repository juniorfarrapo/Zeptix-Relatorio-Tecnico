% !TeX spellcheck = pt_BR
\documentclass[pdftex,10pt,a4paper]{article}

\usepackage[brazilian]{babel}
\usepackage[utf8]{inputenc}
\usepackage[pdftex]{graphicx}
\usepackage{amsmath}
\usepackage{amsfonts}
\usepackage{amssymb}
\usepackage{listings}
\usepackage{color}
\usepackage{fullpage}
\usepackage{float}
\usepackage[colorlinks=true]{hyperref}


\renewcommand
\lstlistingname{Algoritmo}
\renewcommand
\lstlistlistingname{Algoritmos}

\renewcommand
\figurename{Figura}

\renewcommand
\contentsname{Índice}

\renewcommand
\refname{Referências}
\newcommand{\HRule}{\rule{\linewidth}{0.5mm}}

\begin{document}


\begin{titlepage}
\begin{center}

% Upper part of the page. The '~' is needed because \\
% only works if a paragraph has started.
\includegraphics[width=0.25\textwidth]{./brasao-ufc}~\\[1cm]

\textsc{\LARGE Universidade Federal do Ceará}\\[1.5cm]

\textsc{\Large Sistemas Operacionais}\\[0.2cm]
\textsc{\large \emph{Professora}: Gisele Azevedo de Araújo Freitas}\\[1.5cm]

% Title
\HRule \\[0.4cm]
{ \huge \bfseries Escalonador de tarefas para ATmega8 \\ Relatório Técnico\\[0.4cm] }
\HRule \\[1.8cm]

% Author and supervisor
\begin{minipage}{0.7\textwidth}
\begin{flushleft} \large
\emph{Autores:}
Carlos David Braga \textsc{Borges} \\ \ \ \ \ \ \ \ \ \ \ \ \ \ Raimundo Farrapo \textsc{Pinto} Júnior \\
\large
\end{flushleft}
\end{minipage}

\vfill

% Bottom of the page
{Sobral - CE \\ 08 de Junho, 2015}

\end{center}
\end{titlepage}
\tableofcontents

\newpage

\section{Introdução}

	\subsection{Conhecimentos Gerais}


\newpage

\begin{thebibliography}{9}

	\bibitem{wikipr}
	  Wikipedia,
	  \emph{Polynomial regression}. \\
	  Disponível em  $<$https://en.wikipedia.org/wiki/Polynomial\_regression$>$ \\
	  Acesso em 25/09/2014
	
%	\bibitem{bookcoppin}
%			  Ben Coppin,
%			  \emph{Artificial Intelligence Illuminated}. \\
%			  Jones and Bartlett Publishers, 2004
	
\end{thebibliography}

\end{document}